\chapter{Fortran Style}

\section{Basic Rules}
\begin{itemize}
\item In each module, submodule, and programs, be sure to declare
  \verb|implicit none|. Subroutines, functions, types, etc., all
  ``inherit'' this from the module containing them.
\item Organize all code to live in a module, submodule, or a
  \verb|program| unit.
\item Each module or \verb|program| is stored in its own file.
\item Magic numbers and literals are forbidden: store them as parameter
  constants in appropriate modules.
\item Each statement gets its own line.
\end{itemize}

\subsection{Use Fortran 2018 Standard}
The latest Fortran standard is Fortran 2018, though not every feature
has been implemented among compilers.

\subsection{Module Structure}
Modules should have the following skeletal structure:
\begin{verbatim}
module my_mod_name
   use other_mod, only: param1, ..., fun1, ...
   use another_mod, only: ...
   implicit none
   private
   ! parameter declarations
   ! variable  declarations
   public :: integrate, renormalization_group_flow, ...

contains

   subroutine integrate(...)
      ! ...
   end subroutine

   ! ...
end module
\end{verbatim}
The \verb|private| clause will make everything private except those
things explicitly appearing in the \verb|public| list.

\subsubsection{Exception: Use utils}
The exception to the rule of explicitly listing what's used from each
imported module: the {\tt utils} module may be used ``as is''. That is,
we can just write ``{\tt use utils}'' and be done.

\section{Naming Conventions}

\begin{itemize}
\item Use snake case --- i.e., lower case letters, separated by
  underscores, like: \verb|speed_of_light|.
\item Be concise, not cryptic.
\item Use lowercase for all Fortran constructs ({\tt do}, {\tt function},
  {\tt module}, etc.)
\item File names end with ``.f90'' suffix, since we're using free-form
  Fortran.\footnote{This \emph{does} matter for some compilers (e.g., DEC,
  Intel, Compaq). For more about this, see Doctor Fortran's ``Source
  Form Just Wants to be Free'' \url{https://stevelionel.com/drfortran/2013/01/11/doctor-fortran-in-source-form-just-wants-to-be-free/}}
\end{itemize}

\subsection{Module names}
Module names should match the file names. For example, ``{\tt module integrator}''
is contained in the file ``{\tt integrator.f90}''.

\subsection{Object-Oriented Code}
When creating an abstraction (a ``class''), the module name should be
something like ``\verb#foo_class#'' and the class should be named just
``\verb#foo#''.

\section{Unit Testing}

There is an amazing lack of stand-alone unit testing libraries in
\FORTRAN/: many depend on additional languages. The only exception
appears to be Fortran-testanything\footnote{\url{https://github.com/dennisdjensen/fortran-testanything}}.

\subsection{Fortran-TAP}
The basic idea is to use the given test harness, which compiles to
\verb|bin/runtests|, and pass it the test suites the harness should run.

\subsubsection{Test Suite}
Using this term loosely, it's a Fortran \verb|program| which is a
sequence of assertions (usually ``\verb|call ok(<bool>)|'' or
``\verb|call is(lhs,rhs)|''). Such suites begin with something like
``\verb|call plan(<number of tests>)|''.

A module ``\verb|src/my/module.f90|'' has a corresponding test suite ``\verb|test/my/module_tests.f90|''.

Each test suite ``\verb|test/my/module_tests.f90|'' is compiled to a
subprogram contained in ``\verb|bin/tests/my/module_tests|''.

\subsubsection{Saving an Artifact}
``Best practices'' of extreme programming takes a test report, treats it
as an artifact, and saves it to a file. These artifacts are accumulated
over time, to track the test coverage and failure rate (and so on).
We can create a ``poor man's artifact'' by simply redirecting the test
runner's output to a file.

This would schematically look like (in Bash):
``\verb|bin/runtests ... 2>&1 tee tests.log|''. This will redirect both
output and error output into the file ``\verb|tests.log|'', \emph{and}
print it out to the screen.

Ostensibly, we could run
``\verb#find bin/tests/ -printf "%p " | bin/runtests 2>&1 tee tests.log#'',
possibly placing the artifacts in a subdirectory, possibly naming them
by the datetime executed (e.g., ``\verb#... tee "$(date -u -In).log"#''
to write to ``\verb#2021-08-09T18:25:07,332774508+00:00.log#'').