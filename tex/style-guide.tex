\chapter{Fortran Style}

\begin{itemize}
\item In each module, submodule, and programs, be sure to declare
  \verb|implicit none|. Subroutines, functions, types, etc., all
  ``inherit'' this from the module containing them.
\item Organize all code to live in a module, submodule, or a
  \verb|program| unit.
\item Each module or \verb|program| is stored in its own file.
\item Magic numbers and literals are forbidden: store them as parameter
  constants in appropriate modules.
\item Each statement gets its own line.
\end{itemize}

Modules should have the following skeletal structure:
\begin{verbatim}
module my_mod_name
   use other_mod, only: param1, ..., fun1, ...
   use another_mod, only: ...
   implicit none
   private
   ! parameter declarations
   ! variable  declarations
   public :: integrate, renormalization_group_flow, ...

contains

   subroutine integrate(...)
      ! ...
   end subroutine

   ! ...
end module
\end{verbatim}
The \verb|private| clause will make everything private except those
things explicitly appearing in the \verb|public| list.

\section{Naming Conventions}

\begin{itemize}
\item Use kebab case --- i.e., lower case letters, separated by
  underscores, like: \verb|speed_of_light|.
\item Be concise, not cryptic.
\end{itemize}