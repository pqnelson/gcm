\chapter{Introduction}


\epigraph{A complex piece of software consists of simple parts and simple
relations between those parts; the programmer's task is to state those
parts and those relationships, in whatever order is best for human
comprehension\dots%---not in some rigidly determined order like top-down or
%bottom-up.
}{``Literate Programming'' (1984)\\ % section J
\textsc{Donald Knuth}} % The Computer Journal, vol. 27


\section{Structure of a General Circulation Model}

A simple general circulation model consists of a ``dynamical core'':
model the atmosphere as a Navier-Stokes fluid, the Earth as a rotating
sphere. This consists of three equations for the conservation of
momenta, another equation for the conservation of mass (for atmosphere),
and some thermodynamic equation of state for describing pressure as
related to density. Atmospheric general circulation models add equations
of radiative transfer to the dynamic core, and some physical
parametrizations for subgrid details.

\subsection{Navier-Stokes Equations}

The conservation of momenta amounts to Newton's second Law
\begin{equation}
  \frac{\materialD\vec{u}}{\materialD t}
  =\frac{-1}{\rho}\nabla p + \nu\nabla^{2}\vec{u} + \vec{F}
\end{equation}
where $\vec{u}$ is the velocity vector field for the atmosphere, $\rho$
its density, $p$ pressure, $\nu$ kinematics viscosity, and $\vec{F}$
external forces like gravity. Here we use the material derivative
\begin{equation}
  \frac{\materialD f}{\materialD t} = \partial_{t}f + (\vec{u}\cdot\grad)f
\end{equation}
which should match the intuition of, if $\vec{x}(t)$ is the position of
a parcel of atmospheric fluid at time $t$, taking the total time
derivative of $f(\vec{x}(t))$ where we note
$\partial_{t}\vec{x}(t)=\vec{u}(t)$.

The conservation of mass may be written in a variety of ways, usually as
\begin{equation}
  \frac{\materialD\rho}{\materialD t}+\rho\grad\cdot\vec{u}=0.
\end{equation}
The atmosphere typically doesn't have a constant density, so this can't
be simplified much further.

For the equation of state, we'll need to invoke thermodynamics.
Fortunately, the atmosphere is close enough to an ideal gas that we can
rely on quite a few results from the literature. The interested reader
may peruse Vallis~\cite{vallis_2017} or Holton and
Hakim~\cite{holton2013dynamicMeteorology} for details.

\subsection{In Rotating Reference Frame}

We can approximate the Earth as a rotating sphere. The angular velocity
vector $\vec{\Omega}$ is roughly constant, and points in the same
direction as the North pole. We relate the time derivative of quantities
as measured in the inertial frame $\D\vec{A}/\D t$ to how it appears to a
rotating reference frame $(\D\vec{A}/\D t)'$ (primes indicate quantities
in the rotating frame) by
\begin{equation}
  \frac{\D\vec{A}}{\D t} = \left(\frac{\D\vec{A}}{\D t}\right)' + \vec{\Omega}\times\vec{A}.
\end{equation}
When applies to the position $\vec{r}$ of a body, we find the velocity
in an inertial frame related to the velocity in a rotating frame
\begin{equation}
  \frac{\D\vec{r}}{\D t} = \left(\frac{\D\vec{r}}{\D t}\right)' + \vec{\Omega}\times\vec{r}
\end{equation}
and acceleration in the rotating frame may be found from
\begin{equation}
  \frac{\D}{\D t}(\vec{v} - \vec{\Omega}\times\vec{r})
  = \frac{\D\vec{v}'}{\D t}+\vec{\Omega}\times\vec{v}'
\end{equation}
where $\vec{v} = \D\vec{r}/\D t$ and $\vec{v}' = (\D\vec{r}/\D t)'$.
After some algebra, we conclude
\begin{equation}
  \frac{\D\vec{v}'}{\D t}
  = \frac{\D \vec{v}}{\D t}
    - 2\vec{\Omega}\times\vec{v}'
    - \vec{\Omega}\times(\vec{\Omega}\times\vec{r}).
\end{equation}
That is, we have to add the Coriolis effect
$-2\vec{\Omega}\times\vec{v}'$ and the centrifugal pseudoforce
$-\vec{\Omega}\times(\vec{\Omega}\times\vec{r})$. We usually sweep the
centrifugal contribution into the external force term.

The conservation of momentum equations in the rotating frame would
become
\begin{equation}
  \frac{\materialD\vec{u}}{\materialD t} + 2\vec{\Omega}\times\vec{u}
  = \frac{-1}{\rho}\grad p + \vec{F}
\end{equation}
where we understand $\vec{u}$ refers to the fluid velocity for the
atmosphere as observed in the rotating frame.

\subsection{Spherical Coordinates in Rotating Frame}

We change coordinates to spherical coordinates, taking $\lambda$ to
describe the longitude (Eastwards angle), $\theta$ the latitude (angular
distance polewards), and $r$ the radial distance from the center of the
Earth. In these coordinates, any scalar quantity $\varphi$ has its
material derivative be
\begin{equation}
  \frac{\materialD\varphi}{\materialD t}
  = \frac{\partial\varphi}{\partial t}
    + \frac{u}{r\cos(\theta)}\frac{\partial\varphi}{\partial\lambda}
    + \frac{v}{r}\frac{\partial\varphi}{\partial\theta}
    + w\frac{\partial\varphi}{\partial r}.
\end{equation}
Care must be taken when computing the viscosity term, because the vector
Laplacian is defined as
\begin{equation}
  \nabla^{2}\vec{A} := \grad(\grad\cdot\vec{A}) - \curl(\curl\vec{A}).
\end{equation}
Only in Cartesian coordinates does it coincide with the familiar Laplacian.

\begin{prop}
  Mass conservation in spherical coordinates is
  \begin{equation}
    \frac{\partial\rho}{\partial t}
    + \frac{1}{r\cos(\theta)}\frac{\partial(u\rho)}{\partial\lambda}
    + \frac{1}{r\cos(\theta)}\frac{\partial}{\partial\theta}(v\rho\cos(\theta))
    + \frac{1}{r^{2}}\frac{\partial}{\partial r}(r^{2}w\rho)
    = 0
  \end{equation}
  where we have $\lambda$ be the latitude coordinate, $\theta$ the
  longitude coordinate, $r$ the vertical/radial coordinate; and
  $\vec{u}=(u,v,w)$ is the velocity pointing North $u$, East $v$, and
  outward $w$.
\end{prop}

\begin{prop}
  Momentum conservation in a rotating frame using spherical coordinates
  are (neglecting the viscosity term)
  \begin{subequations}
    \begin{align}
      \frac{\materialD u}{\materialD t} -\left(2\Omega + \frac{u}{r\cos(\theta)}\right)(v\sin(\theta)-w\cos(\theta))
      &= \frac{-1}{\rho r\cos(\theta)}\frac{\partial p}{\partial\lambda}\\
      \frac{\materialD v}{\materialD t} -\left(2\Omega + \frac{u}{r\cos(\theta)}\right)u\sin(\theta)
      &= \frac{-1}{\rho r}\frac{\partial p}{\partial\theta} \\
      \frac{\materialD w}{\materialD t} - \frac{u^{2}+v^{2}}{r} -
      2\Omega u\cos(\theta)
      &= \frac{-1}{\rho}\frac{\partial p}{\partial r}-g.\label{eq:vertical-rotating-momentum-conservation}
    \end{align}
  \end{subequations}
  The quadratic terms on the left-hand side involving factors of $1/r$
  are usually called ``metric terms'', and those involving factors of
  $\Omega$ are Coriolis terms.
\end{prop}

Neglecting the viscosity term may be justified by observing, in MKS
units, it is typically $10^{-7}$ times smaller than the $\materialD\vec{u}/\materialD t$
term. For more on this, see Holton and
Hakim~\cite{holton2013dynamicMeteorology}, \S2.4 for scale analysis of
momentum equations.

\begin{prop}
  The radius of the Earth $a$ may be approximated as
  \begin{subequations}
    \begin{equation}
      a = \SI{20000/\pi}{\km}
  \end{equation}
  or, to some 44 digits
  \begin{equation}
    a\approx
    \SI{6366.1977236758134307553505349005744813783858}{\km}.%\,2962
  \end{equation}
  \end{subequations}
\end{prop}
\begin{proof}
  Recall from history, in 1791, the meter was defined such that a
  quarter of the Earth's circumference through the North and South
  poles, starting at the equator, would be
\begin{equation}
  (2\pi a)/4 = \SI{1e4}{\km}
\end{equation}
where $a$ is Earth's radius. Modern measurements note the meridional
circumference of the Earth is $\SI{40007.86}{\km}$, so this
historic definition has an absolute error of $\SI{7.86}{\km}$ (making
it a useful approximation). In reality, Earth's equatorial
radius $a_{eq}\approx \SI{6378.137}{\km}$ and polar radius
$a_{pol}\approx\SI{6356.752}{\km}$ have a geometric mean of
$\SI{6367.44}{\km}$.
\end{proof}
\begin{cor}
  The inverse of Earth's radius is
  \begin{subequations}
  \begin{equation}
    \frac{1}{a} = \frac{\pi}{\SI{2e4}{\km}}
  \end{equation}
  or, to 36 digits
  \begin{equation}
    \frac{1}{a}\approx \SI{1.57079632679489661923132169163975144e-4}{\per\km}
  \end{equation}
  \end{subequations}
\end{cor}

\subsection{Primitive Equations}

There are three simplifications we could make: the shallow fluid
approximation, the hydrostatic approximation, and the traditional
approximation.\footnote{My guide for this material is drawn
principally from Vallis~\cite[\S\S2.2.4--2.2.5]{vallis_2017}, who cites
Phillips~\cite{phillips1966TheEquationsofMotionforaShallowRotatingAtmosphereandtheTraditionalApproximation}
and White~\cite{white2002}.}

We usually simplify these equations further, noting that the height of
the atmosphere is no greater than $\SI{100}{\km}$ which is about
60-times smaller than the radius of the Earth. So we could think of the
atmosphere as a shallow fluid.\marginpar{(1) Shallow fluid: $r=a+z\approx a$,\\
  $\partial/\partial r=\partial/\partial z$}
That is to say, we treat $r=a+z$ where
$a$ is the radius of the Earth and $z\ll a$ the altitude. Consequently,
we replace $\partial/\partial r$ by $\partial/\partial z$ and all other
occurrences of $r$ by $a$. If we make this simplification, then we
are also forced to make the hydrostatic
approximation: in the vertical
momentum equation, the gravitational term is balanced by the pressure
gradient term (and these are the dominant terms in the equation, the
other terms are negligible in comparison). That is to say, we use\marginpar{(2)Hydrostatic Approximation}
\begin{equation}
  \frac{\partial p}{\partial z}=-\rho g
\end{equation}
instead of Eq \eqref{eq:vertical-rotating-momentum-conservation}. These
two approximations are bundle deal --- they are taken together, or not
at all. For large-scale atmospheric (and oceanic) flow on Earth, they're
a pretty good approximation.

The traditional approximation\marginpar{(3) Traditional approximation} discards Coriolis terms in the horizonal
momentum equations, and the smaller metric terms $uw/r$ and
$vw/r$. This approximation is largely independent of the shallow fluid
and hydrostatic approximations, and may be empirically justified by
examining the scale of the various terms.

\begin{prop}
Taking all three approximations gives us the conservation of momentum equations:
\begin{subequations}
  \begin{align}
    \frac{\materialD u}{\materialD t}
    - 2\Omega v\sin(\theta) - \frac{uv}{a}\tan(\theta)
    &= \frac{-1}{\rho a\cos(\theta)}\frac{\partial p}{\partial\lambda}\\
    \frac{\materialD v}{\materialD t}
    - 2\Omega u\sin(\theta) - \frac{u^{2}\tan(\theta)}{a}
    &= \frac{-1}{\rho a}\frac{\partial p}{\partial\theta} \\
    0 &= \frac{-1}{\rho}\frac{\partial p}{\partial z}-g.
  \end{align}
\end{subequations}
\end{prop}

\begin{note}
In our approximations, we now have
\begin{equation}
  \frac{\materialD}{\materialD t}=\frac{\partial}{\partial t} +
  \frac{u}{a\cos(\theta)}\frac{\partial}{\partial\lambda} +
  \frac{v}{a}\frac{\partial}{\partial\theta} + w\frac{\partial}{\partial z}.
\end{equation}
\end{note}

\begin{defn}
It is conventional to introduce the \define{Coriolis parameter}
\begin{equation}
  f = 2\Omega\sin(\theta)
\end{equation}
to simplify calculations. Empirically, $\Omega=7.2921150\times10^{-5}\,\mathrm{rad}\cdot\mathrm{s}^{-1}$.
\end{defn}


\begin{prop}
The mass conservation equation for a shallow atmosphere becomes
\begin{equation}
  \frac{\partial\rho}{\partial t}
  + \frac{u}{a\cos(\theta)}\frac{\partial\rho}{\partial\lambda}
  + \frac{v}{a}\frac{\partial\rho}{\partial\theta}
  + w\frac{\partial\rho}{\partial z}
  + \rho\left(\frac{1}{a\cos(\theta)}\frac{\partial u}{\partial\lambda}
              + \frac{1}{a\cos(\theta)}\frac{\partial(v\cos(\theta))}{\partial\theta}
              + \frac{\partial w}{\partial z}
       \right)
  = 0
\end{equation}
or equivalently
\begin{equation}
  \frac{\partial\rho}{\partial t}
  + \frac{1}{a\cos(\theta)}\frac{\partial(u\rho)}{\partial\lambda}
  + \frac{1}{a\cos(\theta)}\frac{\partial(v\rho\cos(\theta))}{\partial\theta}
  + \frac{\partial(w\rho)}{\partial z}
  = 0.
\end{equation}
\end{prop}

\subsection{Thermodynamics for Atmospheric Fluid}
We need some way to relate density to pressure, which requires
thermodynamics. Atmosphere is approximately an ideal gas, so we could
use
\begin{equation}
  p = \rho R T
\end{equation}
where $R=\SI{287}{\joule\per\kilogram\per\kelvin}$ is the gas
constant for dry air, $T$ is temperature (an unknown). We need another
equation to fix $T$. A first approximation simply fixes $T=T_{0}$ to be
constant, a second approximation fixes $T=T_{0}-T_{1}z$ or something
similar. But realistically, we want to model temperature from first
principles.

For a simple ideal gas, the heat capacity is constant, so
\begin{equation}
  I = cT
\end{equation}
where $c$ is a constant and $I$ satisfies the fundamental thermodynamic
relation
\begin{equation}
  \D I = T\,\D\eta - p\,\D\alpha + \mu\,\D S
\end{equation}
where $I$ is the internal energy of the fluid, $\alpha=1/\rho$ is the
specific volume, $\eta$ is the specific entropy, $\mu$ is the chemical
potential, and $S$ is the composition of the fluid in terms of
``species'' of molecules. If $\dot{Q}$ is the heating applied to the
fluid and $\dot{Q}_{E} = \dot{Q} + \mu\dot{S}$ is the total rate of
energy change, then the internal energy equation
\begin{equation}
  \frac{\materialD I}{\materialD t} + \frac{p}{\rho}\grad\cdot\vec{u} = \dot{Q}_{E}
\end{equation}
or the entropy equation
\begin{equation}
\frac{\materialD \eta}{\materialD t} = \frac{1}{T}\dot{Q}
\end{equation}
could be used to fix the remaining parameters, provided we have a
boundary condition giving us $\dot{Q}$ and $\dot{S}$. For an ideal dry
gas (like the atmosphere) we have
\begin{equation}
  \D I = c_{v}\,\D T
\end{equation}
which gives us
\begin{equation}
c_{v}\frac{\materialD T}{\materialD t} + p\alpha\grad\cdot\vec{u} = \dot{Q}.
\end{equation}
Arguably, for a steady state, $\dot{Q}=0$. Realistically, the Sun
\emph{is} an external heat source, and we need to model how it heats the
atmosphere, which is the rich subject of radiative transfer.

\begin{rmk}
  We can combine several results from thermodynamics to write the first
  law as
  \begin{equation}
    \dot{Q} = c_{p}\frac{\materialD T}{\materialD t} - \frac{1}{\rho}\frac{\materialD p}{\materialD t}
  \end{equation}
  where for the atmosphere
  $c_{p}=1003\,\mathrm{J}\cdot\mathrm{kg}^{-1}\cdot\mathrm{K}^{-1}$, and
  $\dot{Q}$ is the heat rate per unit mass. This is typically used in
  numerical weather prediction. See chapter 2 of
  Kalnay~\cite{kalnay_2002} but be warned: Kalnay uses $Q$ whereas we
  use $\dot{Q}$.
\end{rmk}

\section{Radiative Transfer}

We need to incorporate heating from sunlight, which requires radiative
transfer --- for a review, see Liou~\cite{liou2002introduction} and
Stamnes, Thomas, and Stamnes~\cite{stamnes_thomas_stamnes_2017}.
The latter are part of the team who coded the \DISORT/\index{\DISORT/}
library\footnote{\DISORT/ homepage \url{http://www.rtatmocn.com/disort/}},
which underpins many radiative transfer libraries like the
\RRTM/\index{\RRTM/} suite\footnote{\url{http://rtweb.aer.com/}}.

The basic radiative transfer equation is an integro-differential
equation
\begin{equation}\label{eq:integro-diff-radiative-transfer}
  \mu\frac{\D I(\tau,\mu,\phi)}{\D\tau}
  = I(\tau, \mu, \phi) - S(\tau, \mu, \phi)
\end{equation}
where $\tau$ is the optical depth of the medium; $I$ the diffuse
specific intensity in a cone of unit solid angle along direction $\mu$,
$\phi$; and $S$ is the ``source function''
\begin{equation}
  \begin{split}
    S(\tau, \mu, \phi) &= Q^{\text{(beam)}}(\tau,\mu,\phi)  + Q^{\text{(therm)}}(\tau) \\
    &\qquad + \frac{\omega(\tau)}{4\pi}
  \int^{2\pi}_{0}\int^{1}_{-1} P(\tau,\mu,\phi;\mu',\phi')I(\tau,\mu',\phi')\,\D\phi'\,\D\mu'.
  \end{split}
\end{equation}
Here $\omega(\tau)$ is the ``single scattering albedo'' of the medium
(i.e., the fraction of an incident beam which is scattered by that
volume), and $P$ is the ``scattering phase function'' describing the
angular scattering pattern of an infinitesimal volume. The curious
reader may profitably read chapters 5--8 of Stamnes and
friends~\cite{stamnes_thomas_stamnes_2017}.

\begin{note}
Laszlo and friends~\cite{Laszlo2016} explain how \DISORT/ solves Eq
\eqref{eq:integro-diff-radiative-transfer} in three basic steps:
\begin{enumerate}
\item Transform the equation into a set of radiative transfer equations
  by separation of azimuth dependence;
\item Transforming these equations into a system of ordinary
  differential equations;
\item Solving the System of ODEs using linear algebra.
\end{enumerate}
This strategy may be found in Chandrasekhar's book
\emph{Radiative Transfer} (1960) and refined over time.
\end{note}


\section{ADT Calculus}

Following scientific programming tradition from SAGA\footnote{\url{https://www.ii.uib.no/saga/index.html}},
Rouson and friends~\cite{rouson2011scientific,rouson2008grid}, and much
hard earned experience, we will be encoding mathematical equations
directly into the abstract data types.

% There are a lot of people observing sin(x) is inaccurate on Intel CPUs
% - https://hackernoon.com/help-my-sin-is-slow-and-my-fpu-is-inaccurate-a2c751106102
% - https://randomascii.wordpress.com/2014/10/09/intel-underestimates-error-bounds-by-1-3-quintillion/
% - http://notabs.org/fpuaccuracy/
%
% Also see:
% - https://latkin.org/blog/2014/11/09/a-simple-benchmark-of-various-math-operations/
