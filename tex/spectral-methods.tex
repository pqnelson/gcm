\chapter{Spectral Methods}

Following Durran~\cite{durran2010numerical}, we will discuss spectral
methods to solve differential equations. The intuition guiding these
methods should be thinking of a truncated Fourier series
\begin{equation}
  \phi_{N}(x) = \sum^{N}_{k=-N}c_{k}\E^{\I kx}
\end{equation}
then trying to find the coefficients $c_{k}$ such that they minimize
some error, or satisfy some system of equations. More generally, instead
of working with the basis functions $\exp(\I kx)$ we could use some
other basis: this is what spectral methods study.

When studying a differential equation of the form
\begin{equation}
\frac{\partial\psi}{\partial t} + F(\psi) = 0
\end{equation}
we take the truncated series expansion
\begin{equation}
  \psi = \sum^{N}_{k=1}a_{k}(t)\varphi_{k}(x)
\end{equation}
where $\varphi$ are some family of known functions or polynomials. Note
the arguments of the coefficients and basis functions. The
residual would be
\begin{equation}
  R(\psi) = \frac{\partial\psi}{\partial t} + F(\psi).
\end{equation}
How do we determine the coefficients $a_{k}(t)$?

One strategy seeks to minimize the residual.
We seek the coefficients such that
\begin{equation}
  \|R(\psi)\|_{2}^{2} = \int|R(\psi(x))|^{2}\,\D x
\end{equation}
is minimized. This is one strategy.

A second competing strategy, called \define{Collocation}, requires the
residual to be zero at a discrete set of grid points:
\begin{equation}
  R(\psi(j\Delta x)) = 0\quad\mbox{for all }j=1,\dots,N.
\end{equation}

The third strategy, called the \define{Galerkin approximation},
requires the residual be orthogonal to each of the expansion functions
\begin{equation}
  \langle R, \varphi_{k}\rangle = \int R(\psi(x))\varphi_{k}(x)\,\D x = 0
\end{equation}
for each $k=1,\dots, N$.

\section{Spherical Harmonics}
